\documentclass[a4paper, 11pt, oneside]{article}

\usepackage[utf8]{inputenc}
\usepackage[T1]{fontenc}
\usepackage[french]{babel}
\usepackage{array}
\usepackage{shortvrb}
\usepackage{listings}
\usepackage[fleqn]{amsmath}
\usepackage{amsfonts}
\usepackage{fullpage}
\usepackage{enumerate}
\usepackage{graphicx}
\usepackage{subfigure}
\usepackage{alltt}
\usepackage{url}
\usepackage{indentfirst}
\usepackage{eurosym}
\usepackage{listings}
\usepackage{titlesec, blindtext, color}
\usepackage[table,xcdraw,dvipsnames]{xcolor}
\usepackage{a4wide}
\usepackage{amsmath}
\usepackage{verbatim}
\usepackage{array}
\usepackage{tikz}
\usepackage{float}
\usetikzlibrary{trees}
\usepackage{clrscode3e}

%%%% Page de garde %%%%

\title{INFO-2050 : Mise en page automatique d'une bande dessinée\\Rapport}
\author{Maxime GOFFART \\180521 \and Olivier JORIS\\182113}
\date{2019 - 2020}

\begin{document}
\maketitle
\newpage

\setcounter{section}{3}

\subsection{Approche exhaustive}

\subsection{Formulation récursive de \proc{C}(i, j)}

\proc{C}(i, j)= $ \left\{
	\begin{array}{ll}
        \text{ 0 \textit{si} i $\geq$ n \textit{et} j $\geq$ m}\\
        \proc{E}(i, j) $ + $ \text{min}\{ \proc{C}(i, j-1), \proc{C}(i+1, j-1), \proc{C}(i-1,j-1)\}\textit{ sinon}.
    \end{array}
\right.$

\subsection{Graphe des appels récursifs}

\subsection{Pseudo-code du calcul du coût du sillon optimal}

\subsection{Pseudo-code pour renvoyer l'image réduite de k pixels en largeur}

\subsection{Complexité en temps et en espace}



\end{document}
